% Module 1: The history of Tor & Onion Routing.
\documentclass{beamer}
\usepackage[utf8]{inputenc}
\beamertemplatenavigationsymbolsempty

\usetheme{Warsaw}
\definecolor{mypurple}{rgb}{.49,0,98}
\setbeamercolor*{palette primary}{use=structure,fg=white,bg=mypurple}

\title[Module 1: The history of Tor]
{Module 1: The history of Tor}

\subtitle{The history of Tor and Onion Routing}

\author[]
{C.~Childs\inst{1} \and K.~Misata\inst{2}}

\institute[The Tor Project]
{
  \inst{1}
  Support coordinator / Translation coordinator\\
  The Tor Project
  \and
  \inst{2}
  Outreach coordinator\\
  The Tor Project
}

\logo{\includegraphics[height=1.5cm]{pictures/onion.jpg}}

\begin{document}
\frame{\titlepage}

\begin{frame}
\frametitle{History of the Tor Project}
	\begin{block}{History of the Tor Project}
          In 2002, Tor's founders Roger Dingledine and Nick Matthewson were students at MIT when they were introduced to Paul Syverson of the Naval Research Labs. The three work together on the design, implementation, and deployment of the third-generation onion routing project of the U.S. Naval Research Laboratory. The primary purpose of protecting government communications. Today, it is used every day for a wide variety of purposes by normal people, the military, journalists, law enforcement officers, activists, and many others.
	\end{block}
\end{frame}

\begin{frame}
\frametitle{History of the Tor Project}
	\begin{block}{History of the Tor Project}
		\begin{itemize}
			\item<1-> Community Formed
			\item<2-> Global Need Developed
			\item<3-> Fueled by over 4000 volunteers
			\item<4-> Open Source technology which relies on the work and research of the larger Tor community to keep it viable for the world.
		\end{itemize}
	\end{block}
\end{frame}

\begin{frame}
\frametitle{History of the Tor Project}
	\begin{block}{Three points to remember}
          Three Points to Remember - this may be used as a quiz: \pause
		\begin{itemize}
			\item<1-> Tor is not an acronym - through the technology is based on the onion router, The Tor Project is not an acronym for The Onion Router. This is often misrepresented in the media. Tor is our name with the first letter only being capitalized \pause
			\item<2-> Tor's Technology is Free - as mentioned previously, The Tor Project's software is free software \pause
			\item<3-> Tor's Network is Run by Volunteers
		\end{itemize}
	\end{block}
\end{frame}

\begin{frame}
\frametitle{History of the Tor Project}
	\begin{block}{What is Onion Routing Technology?}
          Onion Routing prevents the transport medium from knowing who is communicating with whom -- the network knows only that communication is taking place. In addition, the content of the communication is hidden from eavesdroppers up to the point where the traffic leaves the OR network.
	\end{block}
\end{frame}

\begin{frame}
\frametitle{History of the Tor Project}
	\begin{block}{Tor's Mission}
          Our mission is to be the global resource for technology, advocacy, research and education in the ongoing pursuit of freedom of speech, privacy rights online, and censorship circumvention. We build innovative, sustainable technology solutions to help people take control of their lives and be free. Keeping the doors to freedom of expression open is what Tor does best.
	\end{block}
\end{frame}

\begin{frame}
\frametitle{History of the Tor Project}
	\begin{block}{Who Uses Tor?}
		\begin{itemize}
			\item<1-> Researchers \pause
			\item<2-> Military \pause
			\item<3-> Law Enforcement \pause
			\item<4-> Activists \pause
			\item<5-> Journalists \pause
			\item<6-> Everyday people...
		\end{itemize}
	\end{block}
\end{frame}

\begin{frame}
\frametitle{History of the Tor Project}
	\begin{block}{Who is The Tor Project?}
          Many people often think we are either an enormous organization with offices all over the world or a small group of hackers sitting in a basement somewhere coding. Actually, we have a very diverse team. \pause
		\begin{itemize}
			\item<1-> Technology Experts - our team and community is comprised of one of the most talented experts in technology and digital security there is in the world. \pause
			\item<2-> Advocates - Additionally, we have a countless number of supporters and advocates around the globe who help us not only with our technology, but in getting the word out to people in need.
		\end{itemize}
	\end{block}
\end{frame}

\begin{frame}
\frametitle{History of the Tor Project}
	\begin{block}{Who is The Tor Project?}
          Many people often think we are either an enormous organization with offices all over the world or a small group of hackers sitting in a basement somewhere coding. Actually, we have a very diverse team.
		\begin{itemize}
			\item<1-> Employees, Contractors - Currently we have 9 full-time employees, approximately 30 contract employees. \pause
			\item<2-> VOLUNTEERS! - and over 5,000 contributing volunteers. \pause
			\item<3-> Researchers - we would not be where we are today without our extensive research community committed to staying ahead of the threat curve and ensuring Tor's technology stays cutting-edge.
		\end{itemize}
	\end{block}
\end{frame}

\begin{frame}
\frametitle{History of the Tor Project}
	\begin{block}{How Tor Makes a Difference}
		\begin{itemize}
			\item<1-> Protecting Online Privacy for All
			\item<2-> Partnering with Academics and Research Institutions
			\item<3-> Working with Policy Makers
			\item<4-> Protecting Journalists, Advocates and General Internet Users
			\item<5-> Defeating Censorship
			\item<6-> Fighting Domestic Violence
		\end{itemize}
	\end{block}
\end{frame}

\begin{frame}
\frametitle{History of the Tor Project}
	\begin{block}{FAQ}
		\begin{itemize}
			\item<1-> How Safe is Tor?
			\item<2-> What do you do about the bad actors who use Tor?
			\item<3-> How do you respond to events in the news about Tor?
			\item<4-> Who are the relay operators?
		\end{itemize}
	\end{block}
\end{frame}

\end{document}

