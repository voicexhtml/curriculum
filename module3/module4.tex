\documentclass{beamer}
\usepackage[utf8]{inputenc}
\beamertemplatenavigationsymbolsempty

\usetheme{Warsaw}
\definecolor{mypurple}{rgb}{.49,0,98}
\setbeamercolor*{palette primary}{use=structure,fg=white,bg=mypurple}

\title[Module 4: Digital Privacy Tips]
{Module 4: Digital Privacy Tips and Tricks}

\subtitle{Tips and tricks for maintaining digital privacy}

\author[C, Childs]
{C.~Childs\inst{1} \and K.~Misata\inst{2}}

\institute[Tor Project]
{
  \inst{1}
  Support assistant / Translation coordinator\\
  The Tor Project
  \and
  \inst{2}
  Outreach coordinator\\
  The Tor Project
}

\logo{\includegraphics[height=1.5cm]{pictures/onion.jpg}}

\begin{document}
\frame{\titlepage}

\begin{frame}
\frametitle{Human Factor}
	\begin{block}{Humans in the system}
		When we are discussing digital security, it is important to remember that there are humans in the system and by design we are all flawed.
	\end{block}
\end{frame}

\begin{frame}
\frametitle{Human Factor}
        \begin{block}{What are we putting out in the wild?}
        First, before even thinking about technology solutions being more aware of what you put out there into the wild is important.  Though you may not be doing anything nefarious, someone may still be watching and waiting.  Sharing all parts of our personal and professional lives on social media and the Internet is becoming so common place that people are almost unconscious.  Many strongly believe that what they put on Facebook can only be seen by certain people and that by putting inappropriate content on the Internet with the intention of being "funny" may not be in the long run.
	\end{block}
\end{frame}

\begin{frame}
\frametitle{Human Factor}
        \begin{block}{What technologies are you using?}
		Do your own inventory and look at what technologies you are using and how.  This should offer a glimpse of where security / privacy awareness should or can be increased.
        \end{block}
\end{frame}

\begin{frame}
\frametitle{Human Factor}
        \begin{block}{What technologies are you using?}
        	\begin{itemize}
			\item<1-> Email
			\item<2-> Instant Messaging
			\item<3-> Mobile devices
			\item<4-> Web Browser
		\end{itemize}
	\end{block}
\end{frame}

\begin{frame}
\frametitle{Human Factor}
        \begin{block}{What technologies are you using?}
		When we are talking about communication and in today's world it isn’t just a single bucket - the lines are constantly being blurred.  Consider communications with sources, conducting research, personal contacts, and professional activities - how are they all connected, where do overlaps exist which may create risk, what devices are being used?
        \end{block}
\end{frame}

\begin{frame}
\frametitle{Total security}
        \begin{block}{There is no such thing as total security}
		Though there are many products out there that offer the "best" security or "100\%" security it's critical to remember that nothing can be completely secure.  For all the reasons discussed above - when there are humans in the system there will always be flaws either in design or in use.  Also, human beings are unpredictable and many people spend a lot of energy focusing on the bad actors in the world.  Our recommendation is to educate yourself on where you are vulnerable and adopt either technologies or changes in behavior to offset those risks.
        \end{block}
\end{frame}

\begin{frame}
\frametitle{5 things to remember}
	\begin{block}{Top 5 things to remember}
        	\begin{itemize}
			\item<1-> Anonymity loves company
			\item<2-> Technology is constantly evolving
			\item<3-> Stay balanced	
			\item<4-> Know where your data is
			\item<5-> Use Tor
		\end{itemize}
	\end{block}
\end{frame}

\end{document}

